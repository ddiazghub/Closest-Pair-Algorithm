\section{Resumen}
El presente informe presenta los resultados de optimizar un algoritmo encuentra la pareja más cercana en una lista de puntos aplicando recursividad. Se realizaron experimentos que consisten en la ejecución del programa con conjuntos de datos cuyo tamaño incrementa en potencias de 2 para el algoritmo recursivo y el algoritmo de fuerza bruta o sin optimizar. Los experimentos se repitieron 10 veces y se registó el tamaño del conjunto de datos, las iteraciones y el tiempo de ejecución para ambos algoritmos. Se graficó el promedio de los resultados para cada algoritmo y se comparó su complejidad temporal promedio. Se obtuvo que la complejidad temporal del algoritmo recursivo es mucho menor a la del algoritmo de fuerza bruta, siendo el primero rápido incluso para conjuntos de datos de entrada de gran tamaño.