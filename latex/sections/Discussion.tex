\section{Análisis de Resultados}

En la presente sección se realizará un análisis de los resultados obtenidos de la parte experimental del laboratorio que se presentaron en la sección anterior. Se verificará que estos resultados concuerden con lo esperado para validar el funcionamiento de ambos algoritmos. La gráfica de los resultados del algoritmo de fuerza bruta se puede ver en la figura 3. Lo que se puede destacar principalmente acerca de estos resultados es un incremento de el número de iteraciones que realizar el algoritmo que aumenta de forma cuadrática cuando incrementa el tamaño del conjunto de datos de entrada. El tiempo de ejecución del programa es un poco aleatoreo para conjuntos de datos de entrada pequeños, esto se debe a razones del sistema operativo totalmente ajenas al algoritmo. Para conjuntos de datos de entrada lo suficiente grandes, el tiempo transcurrido incrementa de manera cuadrática similar al número de iteraciones. Por otro lado, la figura 4 contiene la gráfica de los resultados del algoritmo recursivo. Para este algoritmo se puede ver que el número de iteraciones que se realizan incrementa en un orden más similar con el linear al incrementar el tamaño del conjunto de datos de entrada. Estas iteraciones incluyen el proceso de dividir recursivamente el conjunto de datos de entrada en varios subconjuntos, ejecutar el algoritmo de fuerza bruta entre los elementos de cada uno de estos conjuntos y de ejecutar el algoritmo de fuerza bruta entre los elementos de 2 diferentes conjuntos que pueden ser la pareja más cercana. Para conjuntos de datos muy pequeños, la diferencia con respecto al algoritmo de fuerza bruta es despreciable, sin embargo, cuando el tamaño del conjunto de datos incrementa, el algoritmo recursivo presenta un desempeño mucho mejor al del algoritmo de fuerza bruta al reducirse el número de comparaciones que se hacen por cada uno de los elementos a una o dos dependiendo del tamaño del subconjunto de datos. Además de esto, algo que se puede ver en los resultados para cada una de las 10 ejecuciones individuales para el algoritmo de fuerza bruta (Véase el archivo \textit{output/brute\_force.txt}), es que este algoritmo siempre realiza el mismo número de iteraciones para un conjunto de datos del mismo tamaño, no tiene un mejor o peor caso. Por otro lado, para el algoritmo recursivo (Véase el archivo \textit{output/brute\_force.txt}), el número de iteraciones puede variar entre una repetición y otra, por lo cual en este la complejidad temporal no solamente depende del tamaño del conjunto de datos de entrada sino que también depende de la forma de loss datos.