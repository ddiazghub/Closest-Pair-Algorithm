\section{Conclusiones}

De acuerdo con los resultados obtenidos de la parte experimental al laboratorio es posible concluir que subdividir recursivamente el conjunto de datos en conjuntos más pequeños es una gran optimización para el algoritmo de la pareja de puntos más cercanos comparado con el algoritmo de fuerza bruta aplicado a todo el conjunto de datos de entrada. Dividir recursivamente el conjunto de datos cuando este conjunto está ordenado basado en la posición de los puntos, permite lograr que solamente se compare la posición de cada punto con la de los otros puntos que están más cercanos a este, de esta forma se eliminan las comparaciones innecesarias con puntos que se sabe que están alejados entre sí y que no pueden ser la pareja que se busca. Las mejoras en el desempeño del algoritmo se evidencian especialmente cuando el tamaño del conjunto de datos de entrada es un número muy grande, lo anterior se debe a que el algoritmo de fuerza bruta tiene una complejidad temporal promedio en el orden de $N^2$ mientras que el algoritmo recursivo reduce esta complejidad temporal promedio a orden lineal. Los resultados obtenidos para cada algoritmo corresponden al comportamiento esperado según la teoría y validan el correcto funcionamiento de los algoritmos desarrollados.